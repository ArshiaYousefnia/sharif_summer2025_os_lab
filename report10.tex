\documentclass[12pt]{article}
\usepackage{fontspec}
\usepackage{polyglossia}
\setmonofont{Courier New}
\setmainlanguage{farsi}
\setotherlanguage{english}
\newfontfamily\persianfont[Script=Arabic]{XBZar}
\usepackage{graphicx}
\usepackage{geometry}
\usepackage{hyperref}
\geometry{a4paper, margin=2.5cm}
\usepackage{setspace}
\usepackage{url}
\onehalfspacing
\usepackage{titling}
\usepackage{float}
\usepackage{etoolbox}
\usepackage[backend=biber,style=numeric,sorting=none]{biblatex}
%%%%%%%%%%%%%%%%%%%%%%%%%%%%%%%%%%%%%%%%%%%%%%%%%%%%%%%%%%%%%%%%%%%%%%%%%%%%%
\makeatletter
\newcommand{\persiandigit}[1]{%
	\ifcase#1 ۰\or ۱\or ۲\or ۳\or ۴\or ۵\or ۶\or ۷\or ۸\or ۹\fi
}
\DeclareFieldFormat{labelnumber}{\persiandigit{#1}}
\makeatother
%%%%%%%%%%%%%%%%%%%%%%%%%%%%%%%%%
\newcommand{\persianordinal}[1]{%
	\ifcase#1
	\or اول%
	\or دوم%
	\or سوم%
	\or چهارم%
	\or پنجم%
	\or ششم%
	\or هفتم%
	\or هشتم%
	\or نهم%
	\or دهم%
	\or یازدهم%
	\or دوازدهم%
	\or سیزدهم%
	\or چهاردهم%
	\or پانزدهم%
	\or شانزدهم%
	\or هفدهم%
	\or هجدهم%
	\or نوزدهم%
	\or بیستم%
	\else #1\fi
}

\newcommand{\persianordinalpage}{\persianfont\persianordinal{\value{page}}}


%%%%%%%%%%%%%%%%%%%%%%%%%%%%%%%%%%%%%%%%%%%%%%%%%%%%%%%%%%%%%%%%%%%%%%%%%%%%%
\begin{filecontents}{\jobname.bib}
	@online{a1,
		url = {https://www.geeksforgeeks.org/c/c-program-demonstrate-fork-and-pipe/}
	}
\end{filecontents}

\addbibresource{\jobname.bib}

\defbibheading{bibliography}[]{%
	\begin{RTL}
		\section*{مراجع}
	\end{RTL}
}

%%%%%%%%%%%%%%%%%%%%%%%%%%%%%%%%%%%%%%%%%%%%%%%%%%%%%%%%%%%%%%%%%%%%%%%%%%%%%

\begin{document}
	
	% ==============================
	% Title Page
	% ==============================
	\begin{titlepage}
		\centering
		\vspace*{1cm}
		\includegraphics[width=4cm]{sharif.png}\\[1.5cm]
		{\Large\textbf{دانشگاه صنعتی شریف}}\\[0.5cm]
		{\large\textbf{دانشکده‌ٔ مهندسی کامپیوتر}}\\[1.5cm]
		{\Huge\textbf{گزارش کار آزمایشگاه}}\\[0.5cm]
		{\LARGE\textbf{آزمایشگاه سیستم‌های عامل}}\\[2cm]
		
		\textbf{گزارش آزمایش شماره ۱۰}\\
		(آشنایی با درایور‌ها)
		
		\vfill
		\begin{tabular}{rl}
			\textbf{شماره‌ی گروه:} & ۲۰ \\
			\textbf{گروه:} &
			ارشیا یوسف‌نیا (۴۰۱۱۱۰۴۱۵) \\
			& محمدعارف زارع زاده (۴۰۱۱۰۶۰۱۷) \\
			\textbf{استاد درس:} & دکتر بیگی \\
			\textbf{تاریخ:} & تابستان ۱۴۰۴ \\
		\end{tabular}
	\end{titlepage}
	
	% ==============================
	% Persian Ordinal Page Numbering
	% ==============================
	\clearpage
	\setcounter{page}{1}
	\renewcommand{\thepage}{\persianordinalpage}
	
	\tableofcontents
	\clearpage
	\listoffigures
	%\clearpage
	%\listoftables
	
	% ==============================
	% Switch to Persian Digits (۱, ۲, ۳, ...)
	% ==============================
	\clearpage
	\setcounter{page}{1}
	\pagenumbering{arabic}
	\renewcommand{\thepage}{\persianfont\arabic{page}}
	
	
	% ==============================
	% Main Content
	% ==============================
	\section{آزمایش ۱}
	\subsection{نیازمندی‌های اجرا}
	برای ساخت ماژول و بارگذاری و حذف، مراحل زیر را دنبال کنید:
	
	\begin{english}
		\noindent
		\# install required dependencies \\
		sudo apt upgrade \\
		sudo apt install build-essential linux-headers-\$(uname -r) \\
		\# command to build a driver module \\
		make \\
		\# commands to load and remove a driver module \\
		sudo insmode driver.ko \\
		sudo rmmod driver.ko
	\end{english}
	
	\subsection{توضیح کد و روند آزمایش}
	\begin{figure}[H]
		\centering
		\includegraphics[width=\textwidth]{report10-resources/screenshots/1.png}
		\caption{برنامهٔ درایور آزمایش}
		\label{img:1}
	\end{figure}
	\begin{figure}[H]
		\centering
		\includegraphics[width=\textwidth]{report10-resources/screenshots/2.png}
		\caption{دستورالعمل ساخت یا \textenglish{Makefile}}
		\label{img:2}
	\end{figure}
	\begin{figure}[H]
		\centering
		\includegraphics[width=\textwidth]{report10-resources/screenshots/3.png}
		\caption{نتیجهٔ ساخت و استفاده از درایور}
		\label{img:3}
	\end{figure}
	\section{آزمایش ۲}
	\subsection{نیازمندی‌های اجرا}
	مشابه آزمایش قبل:
	
	\begin{english}
		\noindent
		\# install required dependencies \\
		sudo apt upgrade \\
		sudo apt install build-essential linux-headers-\$(uname -r) \\
		\# command to build a driver module \\
		make \\
		\# commands to load and remove a driver module \\
		sudo insmode driver.ko \\
		sudo rmmod driver.ko
	\end{english}
	\subsection{توضیح کد و روند آزمایش}
	\begin{figure}[H]
		\centering
		\includegraphics[width=\textwidth]{report10-resources/screenshots/4.png}
		\caption{برنامهٔ درایور آزمایش}
		\label{img:4}
	\end{figure}
	\begin{figure}[H]
		\centering
		\includegraphics[width=\textwidth]{report10-resources/screenshots/5.png}
		\caption{دستورالعمل ساخت یا \textenglish{Makefile}}
		\label{img:5}
	\end{figure}
	\begin{figure}[H]
		\centering
		\includegraphics[width=\textwidth]{report10-resources/screenshots/6.png}
		\caption{ساخت و بارگذاری ماژول و تبادل بسته با اینترنت}
		\label{img:6}
	\end{figure}
	\begin{figure}[H]
		\centering
		\includegraphics[width=\textwidth]{report10-resources/screenshots/7.png}
		\caption{مشاهده فایل لاگ بسته‌های دیده شده از درایور و پیام‌های کرنل}
		\label{img:7}
	\end{figure}
        
	% ==============================
	% References
	% ==============================
	\newpage
	\begin{LTR}
		\printbibliography[title={مراجع}]
	\end{LTR}

	
\end{document}

